% -*-latex-*- ; declares latex mode for emacs editor
%\setcounter{chapter}{1}

%https://tex.stackexchange.com/questions/89574/language-option-supported-in-listings
\definecolor{lightgray}{rgb}{.9,.9,.9}
\definecolor{darkgray}{rgb}{.4,.4,.4}
\definecolor{purple}{rgb}{0.65, 0.12, 0.82}

\lstdefinelanguage{JavaScript}{
  keywords={typeof, new, true, false, catch, function, return, null, 
  catch, switch, var, if, in, while, do, else, case, break},
  keywordstyle=\color{blue}\bfseries,
  ndkeywords={class, export, boolean, throw, implements, import, this},
  ndkeywordstyle=\color{darkgray}\bfseries,
  identifierstyle=\color{black},
  sensitive=false,
  comment=[l]{//},
  morecomment=[s]{/*}{*/},
  commentstyle=\color{purple}\ttfamily,
  stringstyle=\color{red}\ttfamily,
  morestring=[b]',
  morestring=[b]"
}

\lstset{
   language=JavaScript,
   backgroundcolor=\color{lightgray},
   extendedchars=true,
   basicstyle=\footnotesize\ttfamily,
   showstringspaces=false,
   showspaces=false,
   numbers=left,
   numberstyle=\footnotesize,
   numbersep=9pt,
   tabsize=2,
   breaklines=true,
   showtabs=false,
   captionpos=b
}


\chapter{Compiler Overview}

\section{Introduction}

The journey begins as Frodo and Bilbo seek out to destroy... At this point, we're sure many readers are familiar with the story of the \textit{Lord of the Rings} and will not go into further detail.\newline

Our story on the other hand begins as two ambitious coders attempting to write a compiler. Should our readers be familiar with the \textit{Dragon Book}, our journey is highly similar in its regard in conquering a dragon, or in this case, the successful completion of a working compiler.

\section{Overview of the Source Compiler}

We shall begin with an overview of our inspiration and the start of an exciting journey towards developing a compiler that compiles Source to Register Machine (RM) language. Unlike a specific machine language (such as ASM, ARM), we aim to use a universal controller data path. 
Essentially, we aim to construct something like this:

\begin{center}
\begin{picture}(80,80)(40,-40)
\setlength{\unitlength}{2pt}
\put(0,-10){\compiler[20]{Source}{RM}{Source}}
\end{picture}
\end{center}
\hspace{15mm}

\noindent
As seen in the above diagram, we aim to compile Source (from-language) to RM (to-language). Additionally, what makes our compiler interesting is that unlike traditional compilers that are written in other languages different from the from and to language, ours is written using the from-language, Source.

\noindent
Let's say we wish to compile a Source program called \textit{gcd}:

\vspace{5mm}
\begin{center}
\begin{picture}(160,40)(0,0)
\setlength{\unitlength}{1pt}
\put(0,0){\program[10]{\texttt{gcd}}{Source}}
\put(40,0){\compiler[10]{Source}{RM}{Source}}
\put(60,-60){\interpreter[10]{Source}{x86}}
\put(60,-80){\machine[10]{x86}}
\put(120,0){\program[10]{\texttt{gcd}}{RM}}
\end{picture}
\end{center}
\vspace{4cm}

\noindent
The compiler takes in a Source program and converts the program to RM. Understandably, our compiler just takes in an input and produces a corresponding output. It does not perform any evaluation of the program. Which is why, we require an RM simulator. The RM simulator does this:

\vspace{5mm}
\begin{center}
\begin{picture}(160,40)(0,0)
\setlength{\unitlength}{1pt}
\put(0,0){\program[10]{\texttt{gcd}}{RM}}
\put(0,-40){\interpreter[10]{RM}{Source}}
\put(0,-80){\interpreter[10]{Source}{x86}}
\put(0,-100){\machine[10]{x86}}
\end{picture}
\end{center}
\vspace{4cm}

\noindent
As observed, the interpreter takes in and evaluates the RM program, producing its output.\newline

Having had a better understanding of our motivations in creating a compiler, we can now discuss an overview for our compiler.